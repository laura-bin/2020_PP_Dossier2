\documentclass{article}

\usepackage[francais]{babel}
\usepackage[T1]{fontenc}
\usepackage{moreverb}       % verbatim with tab

\usepackage{wrapfig}
\usepackage{graphicx}
\usepackage{geometry}
\geometry{hmargin=2.5cm}
\usepackage{amsmath}
\usepackage{siunitx}

\usepackage{graphicx}
\usepackage{subcaption}
\usepackage{float}
\usepackage{hyperref}
\usepackage{setspace}
\usepackage{xcolor}
\usepackage{pdfpages}
\usepackage{enumitem}
\usepackage{lscape}

\usepackage{fancyhdr}       % en-têtes
\usepackage{lastpage}       % numéro de dernière page

\title{}

\title{Programmation procédurale\bigbreak \bigbreak
    \large Dossier 2\bigbreak
    \normalsize Transformations d’images bitmap\bigbreak}
\date{2020 -- 2021}
\author{Laura Binacchi}

\pagestyle{fancy}
\renewcommand\headrulewidth{1pt}
\fancyhead[L]{Laura Binacchi}
\fancyhead[C]{Programmation procédurale}
\fancyhead[R]{\today}


\begin{document}
    \pagenumbering{gobble}
    \includepdf[pages={1}]{pdg}
    \newpage
    \tableofcontents
    \newpage
    \pagenumbering{arabic}

    \section{Cahier des charges}
    \paragraph{}
    On demande d’écrire un programme en C qui permet de générer une série de fichiers image au format .bmp 24 bits. Ces images seront  obtenues par application de filtres décrits ci-dessous à des images existantes. Le programme fonctionnera en 2D, le but de l’exercice n’est pas artistique mais bien de concevoir et implémenter des effets graphiques faisant appel à plusieurs images. Le programme sera décomposé en blocs fonctionnels cohérents, fera appel à l’allocation dynamique des matrices utilisées et sera écrit de manière lisible et bien documentée. Il utilisera au maximum les notions de programmation vues au cours, spécialement les structures, les fichiers binaires, les boucles imbriquées complexes à indices multiples et variables.

    \paragraph{}
    Ce programme sera purement «batch», aucune interactivité n’est demandée. Le programme principal appellera un ensemble de routines de tests générant les images demandées, mais aussi testant de manière systématique les fonctions primitives nécessaires.

    \subsection{Filtre 1: pavage d'images}
    \paragraph{}
    Une image est divisée en pavés carrés. Dans une première version, ces pavés seront de couleur constante, égale à la couleur moyenne de la portion d’image. Ensuite le pavé sera obtenu par redimensionnement de l’image et recolorisation afin de rendre l’aspect global de l’image.

    \subsection{Filtre 2: images en donut}
    \paragraph{}
    Une transformation de coordonnées entre des coordonnées cartésiennes et polaires permet de représenter l’image sous forme d’un donut.
    
    \paragraph{}
    Ces deux effets sont évidement combinables...




    \newpage
    \section{Description du programme}
    \subsection{Manipulation des fichiers bitmap}
    \paragraph{}
    L'implémentation des filtres demande au préalable d'avoir implémenté des fonctions de lecture (chargement d'une image en mémoire RAM) et d'écriture d'un fichier bitmap. Il est également nécessaire de définir la structure du bitmap manipulé : la matrice de pixels composant l'image et les métadonnées du fichier. Ces structures et fonctions utilitaires sont définies dans le fichier \emph{bmp\_file.h} et implémentées dans le fichier \emph{bmp\_file.c}.


    \subsubsection{Métadonnées du fichier bitmap}
    \paragraph{}
    Le programme demandé manipule des fichiers BMP3, i.e. 24 bits : chaque pixel y est représenté par un ensemble de 3 bytes correspondant respectivement aux niveaux de rouge, bleu et vert du pixel. L'image est représentée par une matrice de ces pixels.

    \paragraph{}
    Les fichiers BMP3 contiennent en début de fichier un certain nombre de métadonnées sur le fichier et sur l'image : nombre de lignes et de colonnes de l'image, offset de l'image, etc. La structure de ces métadonnées, définie par la structure \texttt{BITMAPINFOHEADER} de la documentation \emph{Windows} pour la version 3 de la norme \emph{bitmap}, est documentée dans le fichier \emph{bmp\_file.h}.
    
    \paragraph{}
    La fonction \texttt{print\_bmp3\_info} permet d'afficher les métadonnées d'une image chargée en mémoire. Cette fonction permet de vérifier la conformité des données chargées avec l'image source. Elle permet également de vérifier l'intégrité des données chargées avec celles du fichier bitmap en ouvrant ce fichier dans un éditeur hexadécimal.


    \subsubsection{Chargement d'un fichier bitmap}
    \paragraph{}
    La fonction \texttt{load\_bmp3} permet de charger une image bitmap dans la mémoire RAM du programme.
    
    \paragraph{}
    Cette fonction doit charger en deux temps l'identifiant du fichier (les caractères \texttt{BM}) et le reste des données afin d'éviter un décalage dû au padding ajouté par le compilateur. En effet, le compilateur ajoute de l'espace pour que les champs soient alignées sur quatre bytes, ce qui ne pose de problème que dans le cas de la lecture des informations du fichier.
    
    \paragraph{}
    Le chargement de l'image dans la matrice de pixels tient également compte du padding du fichier bitmap : chaque ligne est composée d'un certain nombre de pixels de trois bytes mais la taille de la ligne entière est un multiple de quatre bytes.
    
    \paragraph{}
    Dès que l'image n'est plus utilisée, la fonction \texttt{free\_bmp3} doit être appelée pour libérer la mémoire allouée à la matrice de pixels de l'image.



    \subsubsection{Écriture d'une image}
    \paragraph{}
    La fonction \texttt{write\_bmp3} écrit une image en mémoire dans un nouveau fichier (ou en écrasant le fichier existant s'il y en avait un). Comme la fonction de chargement, elle tient compte du padding des lignes du fichier bitmap.

    \subsubsection{Création d'une image}
    \paragraph{}
    La fonction \texttt{create\_bmp3} permet de créer une nouvelle image aux dimensions souhaitées. Par défaut, l'image est noire car sa matrice de pixels est initialisée à \texttt{0}. Il est possible de la remplir avec une autre couleur en appelant la fonction \texttt{fill\_bmp3}. Un certain nombre de couleurs de bases sont définies dans le tableau \texttt{colors}. 



    \subsection{Pavage d'images}
    \paragraph{}
    Le pavage d'images est implémenté dans le fichier \emph{mosaic\_filter.c}. Ce filtre utilise la fonction utilitaire \texttt{reduce} qui permet, d'une part, de générer une image réduite contenant les couleurs moyennes des tuiles de l'image filtrée et, d'autre part, de générer l'image réduite qui sera insérée dans la tuile.

    \paragraph{}
    Chaque tuile de la mosaïque est crée à partir de l'image réduite en tuile combinée à la couleur moyenne gardée en mémoire dans l'image réduite pour créer un effet de transparence. La proportion de couleur gardée de chacune des deux images est calculée à partir d'un facteur de transparence passé en paramètre à la fonction. Ce facteur varie entre \texttt{0} et \texttt{10} compris : à \texttt{0}, l'image insérée n'est pas du tout transparente et recouvre donc totalement la couleur de la tuile et à \texttt{10} elle est complètement transparente et donc invisible.

    \paragraph{}
    Cette fonction pourrait encore être améliorée pour pouvoir s'appeler récursivement afin d'ajouter plus d'un niveau d'insertion : la mosaïque pourrait par exemple être composée de tuiles elles-mêmes composées de tuiles contenant l'image d'origine. Cet effet peut être testé en appelant deux fois la fonction de filtre sur une image mais l'image de base est difficilement reconnaissable quand le filtre est appliqué deux fois.
    

    \subsection{Images en donut}
    \paragraph{}
    Le filtre donut est implémenté dans le fichier \emph{donut\_filter.c}. Il permet de transformer un fichier BMP3 chargé en mémoire en un donut de rayon intérieur donné (en pixels) et d'étendue donnée (en pourcentage).

    \paragraph{}
    La taille de l'image "donut" est définie par rapport à l'espace occupé par le donut produit par le filtre selon les paramètres donnés, i.e. si la portion de donut est plus petite qu'un donut complet, les bords sont coupés.

    \paragraph{}
    La correspondance entre les pixels du donut et les pixels de l'image d'origine est établie selon les formules trigonométriques fournies au cours. Ces calculs de correspondance peuvent produire des erreurs d'arrondis qui font qu'il manque certains pixels sur les bords de l'image. L'implémentation du filtre donut corrige les arrondis pour que le bord extérieur du donut soit bien présent, ce qui peut générer des erreurs d'un ou deux pixels dans le calcul du rayon interne.



    \newpage
    \section{Tests}
    \paragraph{}
    Les tests sont implémentés dans le dossier \emph{tests} sous forme de routines appelant les différentes fonctions utilisées par le programme. Ils sont compilés en fichiers exécutables par un \emph{Makefile} généré par \emph{cmake}. La commande \texttt{cmake ..}\footnote{Pour ajouter des informations sur le code source lors du développement, j'utilise la commande \texttt{cmake -DCMAKE\_C\_FLAGS=-g ..} qui permet de remonter les erreurs plus facilement avec \emph{gdb}. Pour la version finale, j'utilise la commande \texttt{cmake -DCMAKE\_BUILD\_TYPE=Release ..}} lancée depuis le répertoire \emph{build} génère le \emph{Makefile}. La commande \texttt{make} lancée depuis ce même répertoire génère les différents exécutables.


    \subsection{Chargement d'une image}
    \paragraph{}
    Le chargement d'une image en mémoire RAM est testé dans le fichier \emph{test\_load.c}. Il teste le chargement en mémoire RAM des images de test placées dans le répertoire \emph{images} et affiche les données du header pour vérifier le chargement correct de celui-ci. Ces informations permettent notamment de vérifier que les hauteurs et largeurs en pixels des images et les tailles en bytes correspondent bien aux données des images testées.

    \subsection{Copie d'une image}
    \paragraph{}
    Le test de copie d'image implémenté dans le fichier \emph{test\_copy.c} permet de tester la fonction d'écriture d'une image bitmap en mémoire dans un fichier bitmap. Il permet également de vérifier l'intégrité des données chargées dans la matrice de pixels. Les images crées sont comparées aux images d'origine avec la commande \texttt{diff}.


    \subsection{Création d'une image}
    \paragraph{}
    La création d'une nouvelle image est testée par le fichier \emph{test\_create.c}. Ce test vérifie que la fonction de création crée bien une image noire (la matrice de pixels est initialisée à \texttt{0}) aux dimensions demandées. Il crée des images de largeur variables dont le modulo \texttt{4} est égal à \texttt{0}, \texttt{1}, \texttt{2} et \texttt{3} pour vérifier que l'écriture de l'image gère correctement le padding des lignes. Il permet également de tester la fonction de remplissage d'une matrice de pixels avec une couleur donnée.


    \subsection{Filtre de pavage}
    \paragraph{}
    La fonction utilitaire de réduction d'une image utilisée pour le filtre de pavage est testée dans les fichiers \emph{test\_reduction.c} et \emph{test\_tile.c} qui testent les deux manières dont la fonction sera utilisée par le filtre. Le fichier \emph{test\_reduction.c} teste également les cas d'erreurs de la fonction de réduction.

    \paragraph{}
    Le test de réduction génère une image réduite d'un facteur donné. Chacun de ses pixels est initialisé avec la couleur moyenne de l'ensemble des pixels de l'image source qui y sont réduits. Les réductions créées à partir des moyennes de couleurs sont plus lisible (i.e. plus proche de l'image d'origine) que les réductions ne reprenant qu'un pixel sur \texttt{x} de l'image d'origine. Cet effet est particulièrement visible pour le fichier \emph{landscape1.bmp} où les bleus sont atténués mais ne sont pas perdus.

    \paragraph{}
    Le test de création d'une tuile génère une tuile de taille de donnée et non plus de facteur de réduction donné. La tuile à insérer dans l'image pavée sera toujours carrée. Ce test permet de vérifier le bon fonctionnement des offsets : les tuiles sont bien générées à partir du centre de l'image et pas des bords.

    \paragraph{}
    Le filtre de pavage en tant que tel est testé par le fichier \emph{test\_mosaic.c}. La création de la mosaïque est testée avec différentes tailles de tuiles et différentes transparences. Le dernier test réapplique le filtre de pavage à une image déjà passée par le filtre.


    \subsection{Filtre donut}
    \paragraph{}
    Le filtre donut est testé par le fichier \emph{test\_donut.c}. Il teste la génération de donuts de différentes étendues et de différents rayons internes. Les images sont produites avec un fond qui contraste bien avec l'image d'origine pour distinguer clairement le donut généré par le filtre. Les images choisies permettent de voir l'effet de distorsion du filtre sur les lignes.

    \paragraph{}
    Ce fichier teste également les cas d'erreurs de la fonction de filtre.



    \subsection{Valgrind}
    \paragraph{}
    Tous les exécutables ont été testés avec \emph{valgrind} pour vérifier que tous les espaces de mémoire alloués lors de l'exécution du programme sont bien libérés. Ces tests sont effectuées avec la commande \texttt{valgrind --leak-check=full -s ./test}\footnote{\texttt{leak-check=full} pour avoir des détails sur les fuites mémoires et \texttt{-s} pour avoir des détails sur les erreurs} où \texttt{test} est le nom de l'exécutable à tester.

    \paragraph{}
    Ces tests m'ont permis de rester attentive à toujours libérer les espaces mémoire alloués aux matrices de pixels.



    \newpage
    \section{Problèmes rencontrés}
    \paragraph{}
    La première difficulté de ce travail concerne le format du fichier bitmap3, en particulier le padding des lignes de la matrice de pixels. En cherchant des informations sur la norme bitmap je n'avais d'abord pas vu ce padding et le fait de ne pas l'avoir pris en compte s'est révélé dans les tests de copie d'images. J'aurais pu reprendre les fonctions fournies mais les réimplémenter m'a permis de bien comprendre la structure des fichiers bitmap.

    \paragraph{}
    Sur base de fonctions utilitaires correctement implémentées, je n'ai pas rencontré de difficulté particulière pour l'implémentation du filtre de pavage. J'ai procédé étape par étape en ajoutant petit à petit des paramètres et des fonctionnalités supplémentaires.

    \paragraph{}
    Pour le filtre donut, régler les problèmes d'arrondis a été la plus grande difficulté. Il a fallu trouver où corriger les arrondis et où les imprécisions de calcul étaient plus ou moins gênantes pour le rendu final. J'ai privilégié de ne pas perdre le bord extérieur de l'image en perdant parfois un peu de précision sur le calcul du rayon interne. L'implémentation du filtre est également plus précise sur les images dont l'étendue et supérieure à 50\%.

    \paragraph{}
    Le programme est bien entendu toujours améliorable. J'aurais par exemple pu implémenter le facteur de réduction d'image de la fonction de réduction d'image et le facteur de transparence du filtre de pavage sous forme de pourcentage comme je l'ai fait pour l'étendue du donut.

    \paragraph{}
    J'aurais également pu rendre les filtres plus généraux en les rendant indépendants du type d'image filtrée. Dans ce programme, les filtres s'appliquent spécifiquement à des images bitmap3. C'est bien ce qui est demandé par le cahier des charges mais cela nécessitera un travail supplémentaire pour adapter ces fonctions à un autre type d'image.
    
\end{document}
